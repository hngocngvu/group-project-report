\begin{figure}[H]
    \centering
    \begin{tikzpicture}[
        % 1. Global Styles
        node distance = 1.5cm and 2cm, % Vertical and Horizontal spacing
        >=Stealth,                       % Nice arrow tip style
        line width=1pt,                  % Thickness of arrows
        % Style for the image nodes
        picnode/.style={
            draw=black,       % Border color (optional, remove if not needed)
            inner sep=0pt,    % No padding between image and border
            outer sep=2pt,    % Space for arrows to touch
        }
    ]

    % -----------------------------------------------------------
    % 2. Define Helper Command for Images
    % Change width/height here to control size
    % -----------------------------------------------------------
    \newcommand{\img}[1]{%
        \includegraphics[width=1.5cm, height=2cm, keepaspectratio]{#1}%
    }
    
    % --- LEFT COLUMN ---
    \node[picnode] (img1) {\includegraphics[width=4cm]{img/1.pdf}};
    \node[picnode, below=of img1] (img2) {\includegraphics[width=4cm]{img/2.pdf}};
    \node[picnode, below=of img2] (img3) {\includegraphics[width=4cm]{img/3.pdf}};
    % --- RIGHT COLUMN ---
    % Place Image 5 to the right of Image 1
    \node[picnode, right=of img1] (img5) {\includegraphics[width=4cm]{img/5.pdf}};
    
    % Place Image 4 to the right of Image 3
    \node[picnode, right=of img3] (img4) {\includegraphics[width=4cm]{img/6.pdf}};

    % Place Image 6 
    \node[picnode, below=of img5] (img6) {\includegraphics[width=4cm]{img/4.pdf}};

    % -----------------------------------------------------------
    % 3. Draw Arrows
    % -----------------------------------------------------------
    
    % Left Vertical Flow
    \draw[->] (img1) -- node[left, font=\small] {1. Bubble Segmentation} (img2);
    \draw[->] (img2) -- node[right, font=\small] {3. Convex Defect}(img3);

    % Horizontal Connections
    \draw[->] (img2) -- node[right, font=\small] {2. Mask Overlay} (img5);
    \draw[->] (img3) -- node[above, font=\small] {4. OCR} (img4);

    % Right Vertical Flow (Sandwich)
    \draw[->] (img5) -- node[right, font=\small] {6. Typesetting}(img6); % Down
    \draw[->] (img4) -- node[right, font=\small] {5. Translation}(img6); % Up

    \end{tikzpicture}
    \caption{Pipeline Workflow}
    \label{pipeline}
\end{figure}

Figure~\ref{pipeline} illustrates the overall pipeline of our manga translation system.
First, speech bubbles in the input image are segmented using a YOLO-based model, producing instance-level bounding boxes and original segmentation masks.
These masks are then refined using convex defect analysis to improve the geometric quality of bubble regions and to separate overlapping bubbles.
Next, \texttt{manga-ocr} is applied to recognize Japanese text within the refined speech bubble regions. The recognized text is translated to English using \texttt{elan-mt-ja-en}.
Finally, the translated text is rendered onto the original image based on the original segmentation masks, preserving the visual appearance of the speech bubbles in the final output.