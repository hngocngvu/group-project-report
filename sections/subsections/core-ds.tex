The application state is managed through several data structures.
The ImageProject class acts as the storage for a specific file. It keeps tracks of the file path, the processing status (queued, processing, ready, or error), and contains a list of detected objects (bubbles).
The BubbleData class stores individual text regions. It wraps the logical and geometric data required for both the UI and the backend typesetter:

\begin{enumerate}
    \item Geometry: The polygon (QPolygonF) for UI rendering and the bbox for cropping.
    \item Masks: The raw\_mask required for the painting process.
    \item Text: Stores both the ocr\_text (source) and trans\_text (target).
\end{enumerate}

In the graphical interface, the QGraphicsPolygonItem objects are linked to their corresponding BubbleData instances using Qt's item data storage (setData). This linkage allows the system to retrieve the underlying data object immediately when a user interacts with a visual element on the GUI scene.