Layout analysis and reading order estimation are fundamental to comic understanding.
These topics have attracted considerable research attention, as evidenced by several publications in the literature.
In 2012, Ngo et al. \cite{ngo2012panel} emphasized how graphic elements such as panels and speech balloons affected to full text indexing, therefore, 
proposed a region-based and morphology-based method to extract panels in a comic page. 
Two years later, Pang et al. \cite{pang2014robust} presented a robust method for panel extraction in comics by first closing open panels and identifying background masks.
The panel block is then recursively splitted the panel into sub-blocks from which panel shapes are recovered.
In 2019, Nguyen et al. \cite{nguyen2019comic} had a different approach by considering panel extraction in image segmentation task.
They adopted U-Net architecture to perform multiclass classification for each pixel, whether it belongs to the background, panels or borders.
In 2022, Omori et al. \cite{omori2022comic} proposed algorithms to estimate the reading order of comic frames by sorting and finding rightmost or topmost frame. 
If it exists, the first frame will be chosen according to its position in the sorted list of frames.
If there is no topmost or rightmost frame, the first frame in the sorted output will be considered as the first.
