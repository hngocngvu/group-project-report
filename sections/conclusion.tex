In this work, we present an end-to-end pipeline for automatic Japanese manga translation, integrating speech bubble segmentation, optical character recognition, machine translation, and typesetting into a unified Qt-based desktop application. Our system leverages YOLO family segmentation models for bubble and panel segmentation, MangaOCR for Japanese text recognition with post-processing, and Qwen2.5-1.5B-Instruct fine-tuned with LoRA for context-aware translation.
A limitation of segmentation models is that bubbles with transparent background are failed to detect. 
By assuming that images with a width-to-height ratio greater than 1.0 contain two pages, we can split them for separate processing and merge the results afterward. However, this assumption does not always hold.
For layout analysis, while our approach has proven effective, it is still a naive one. By dividing the image horizontally and grouping panels into different rows, a panel whose height spans across two rows may be misclassified, making the correct reading order harder to follow.\\

Future work includes improving OCR accuracy through expanded training data and enhanced post-processing techniques. For translation, more aggressive LoRA fine-tuning could be applied by increasing the number of training epochs and incorporating higher-quality Japanese–English parallel datasets. The flexible JSON schema provides opportunities to integrate richer contextual information, which can be exploited in the long run.
More sophisticated context-aware translation that captures conversation history and speaker identity across dialogue sequences can be developed. The application interface can be improved to provide a more intuitive experience with additional features tailored to professional translator workflows.
Entire image could be processed as a single unit to avoid errors caused by incorrect page splitting.
To reduce noise in the overall pipeline, future work will also focus on improving the panel and speech bubble ordering algorithms, ensuring a more accurate reading sequence.


