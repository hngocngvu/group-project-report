We integrate three specific variants of the model to balance translation quality and computational efficiency, as shown in
Table \ref{tab:translation-variants}.

\begin{table}[H]
    \centering
    \begin{tabular}{|c|c|l|}
        \hline
        \textbf{Model Name} & \textbf{Parameters} & \textbf{Description} \\
        \hline
        Mitsua/elan-mt-bt-ja-en & 61M & Back-translation augmented variant with \\
        & & improved handling of colloquial expressions\\
        \hline
        Mitsua/elan-mt-base-ja-en & 61M & Base model trained on parallel corpora \\
        \hline
        Mitsua/elan-mt-tiny-ja-en & 15M & Lightweight variant optimized for inference speed \\
        \hline
    \end{tabular}
    \caption{Variants of the Mitsua/elan-mt model family}
    \label{tab:translation-variants}

\end{table}

The \textbf{bt} (back-translation) variant employs data augmentation through synthetic parallel data generation, where monolingual English text is translated back to Japanese to create additional training pairs

The \textbf{tiny} variant reduces the model depth and hidden dimensions, achieving approximately 4× parameter reduction while maintaining reasonable translation quality. This variant is suitable for resource-constrained environments or applications requiring real-time inference.