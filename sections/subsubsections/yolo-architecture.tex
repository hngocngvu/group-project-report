In this study, we employ the YOLO architecture adapted for instance segmentation tasks. Specifically, we investigate two versions: YOLOv8 \cite{10533619} and YOLOv11 \cite{yolo11_ultralytics}.  While both models share a fundamental Anchor-Free, One-Stage detection paradigm, YOLO11 introduces significant architectural refinements to improve feature extraction and processing speed.
The segmentation architecture consists of three primary components: the Backbone, Neck, and Head.

\begin{enumerate}
    \item \textbf{Backbone:} The Backbone is responsible for extracting feature maps from the input manga pages.
YOLOv8 utilises the CSPDarknet structure enhanced with C2f (Cross Stage Partial bottleneck with 2 convolutions) modules. The C2f module improves information flow by merging the concepts of C3 modules \cite{9150780} and ELAN (Efficient Layer Aggregation Networks) \cite{wang2022designingnetwork}.

YOLOv11 replaces the C2f module with the C3k2 block (C3k with a customable kernel size). Furthermore and integrates the C2PSA (Convolutional Block with Parallel Spatial Attention) module at the end of the backbone \cite{yolo11_ultralytics}. This attention mechanism allows the model to focus more effectively on spatial features such as the contours of a bubble against complex manga backgrounds before passing data to the Neck. 
The architectures of the C2PSA module, the attention block in YOLOv11, the C2f module, and the C3k2 module are illustrated in Figures \ref{fig:c2psa_psablock} \cite{hidayatullah2025yolov8_yolo11_review} and \ref{fig:c2f_c3k2}  \cite{hidayatullah2025yolov8_yolo11_review}.

\begin{figure}[H]
    \centering
    \begin{subfigure}{0.48\textwidth}
        \centering
        \includegraphics[width=0.9\linewidth]{img/C2PSA.png}
        \caption{C2PSA module}
    \end{subfigure}
    \hspace{-1.5em}
    \begin{subfigure}{0.48\textwidth}
        \centering
        \includegraphics[width=0.7\linewidth]{img/PSABlock.png}
        \caption{Attention block (YOLOv11)}
    \end{subfigure}
    \caption{Architecture of PSA-based modules: (a) C2PSA module; (b) Attention block used in YOLOv11 \cite{hidayatullah2025yolov8_yolo11_review}}
    \label{fig:c2psa_psablock}
\end{figure}


\begin{figure}[H]
    \centering
    \begin{subfigure}{0.48\textwidth}
        \centering
        \includegraphics[width=0.9\linewidth]{img/C2f.png}
        \caption{C2f module}
    \end{subfigure}
    \hspace{-1.5em}
    \begin{subfigure}{0.48\textwidth}
        \centering
        \includegraphics[width=0.7\linewidth]{img/C3k2.png}
        \caption{C3k2 module (YOLOv11)}
    \end{subfigure}
    \caption{Comparison of backbone modules: (a) C2f module; (b) C3k2 module adopted in YOLOv11 \cite{hidayatullah2025yolov8_yolo11_review}}
    \label{fig:c2f_c3k2}
\end{figure}

    \item \textbf{Neck (Feature Fusion):} Both versions employ a PANet (Path Aggregation Network) \cite{8579011} structure combined with an FPN (Feature Pyramid Network) \cite{8099589}. This component aggregates features from different backbone stages. Since manga bubbles can range from tiny whispers to page-dominating shouts, this multi-scale fusion is critical for detecting instances of all sizes effectively.
    \item \textbf{Segmentation Head for Mask Generation:} Unlike standard object detection, the output of segmentation head is pixel-wise mask, followed by Decoupled Head architecture \cite{ge2021yoloxexceedingyoloseries}, which separates the classification and regression tasks.
    
    Following the YOLACT (You Only Look At Coefficients) principle \cite{9010373}, the mask generation pipeline begins with Protonet, a dedicated branch that learns a fixed set of k prototype masks—generic bases that are shared across every image and never conditioned on any particular object. Concurrently, the detection head predicts, for each anchor, k scalar mask coefficients that encode the instance-specific contribution of every prototype. These coefficients are multiplied with their corresponding prototypes, the weighted maps are summed, a sigmoid activation converts the logits to per-pixel probabilities, and the resulting mask is finally cropped to the predicted bounding box to produce the assembled instance mask.
    The architecture of YOLACT is illustrated in Figure \ref{fig:yolo-segmentation-head}.

    \begin{figure}[H]
        \centering
        \includegraphics[width=0.7\linewidth]{img/YOLACT.png}
        \caption{YOLACT architecture}
        \label{fig:yolo-segmentation-head}
    \end{figure}

\end{enumerate}
